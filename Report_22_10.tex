\documentclass[twocolumn,11pt]{article}

% Packages
\usepackage[utf8]{inputenc}  % UTF-8 encoding
\usepackage{graphicx}        % For figures
\usepackage{amsmath, amssymb}% For math symbols
\usepackage{hyperref}        % Hyperlinks
\usepackage{geometry}        % Page layout
\usepackage{lipsum}          % Dummy text
\usepackage{enumitem}        % Custom lists

\geometry{margin=1in}

% Title and Author
\title{GetStanding: A Web Application for Healthier Desk Habits - Assignment 2 }
\author{Group 8: \\ Piotr Szczesny –piszc24@student.sdu.dk \\ Vlad-Alexandru Ionita -vlion24@student.sdu.dk \\ Karolina Malik -kamal24@student.sdu.dk \\ Bartosz Bobkiewicz -babob24@student.sdu.dk \\ Viktor Barabás -vibar24@student.sdu.dk \\ Azra Talum -aztal24@student.sdu.dk}

\date{\today}

\begin{document}


\maketitle


\twocolumn

\section{Introduction}
\textbf{Motivation:} Prolonged sitting harms both well-being and productivity, yet people often neglect or overlook this aspect of day-to-day life. Additionally, the lack of easy-to-use tools to improve sedentary lifestyles is often one of the main reasons for people not getting a better grasp of their unhealthy habits and makes it difficult to accurately track and assess physical activity, limiting the ability to prevent long-term negative effects. The primary motivation behind this project is to give office workers and students an appropriate tool that can aid them in their journey towards balancing their sedentary and active hours.
\\\\
\textbf{Project:} GetStanding is a web application designed for people who spend most of their day sitting or generally leading an unhealthy lifestyle. The platform promotes a healthier lifestyle by granting users an easy-to-use application which they can use in their journey. Core features include activity tracking, integration with fitness data inputted by the user and general fitness tools such as calorie and meal tracking. GetStanding helps users maintain better posture, stay active, and improve overall well-being throughout their day, as well as their mood.
\\\\
\textbf{Contributions:} The following sections give an insight on the workflow of the application and shows how efficient code can be applied into solving a real-life problem. The project demonstrates how a Model-View-Controller framework should be applied to a project, integrating elements of database handling, API requests and responses, authentication and authorization, and utilization of CRUD.

\section{Backend and Resource Management}
\subsection{Backend}
The backend follows the MVC framework to ensure a well-structured project. The backend techstack uses Express.js and a PostgreSQL database running in Docker.
The three main architecture components are :
\begin{itemize}[noitemsep]
    \item Models handle data, business logic and database queries. They define how data is created, read, updated and deleted.
    \item Views (not part of the backend code but a portion of MVC structure) are responsible for rendering the User interface and sending requests to the backend controllers.
    \item Controllers act as intermediaries between the client and the model. Handling the validation of incoming requests and sending back a properly structured response
 \end{itemize}
Backend includes other responsibilities:\\
Docker provides containerization for the Database ensuring consistent development environments.\\
Routing is in control of mapping URLs to controllers.

\subsection{Resource Management}
\textbf{Resource: WorkoutModel}
\\
Description: WorkoutModel is the data model which holds all the relevant properties regarding a workout. Through CRUD operations it can be managed by the admin.
\\
\begin{itemize}
\item getAll (GET) - Returns all workouts.
\item create (POST) - Adds a new workout using data from the request body.\
\item deleteById (DELETE) - Removes workout by its ID.\\
\end{itemize}

\textbf{Resource: UserModel}
\\
Description: UserModel holds information regarding GetStanding users. CRUD operations are implemented to allow core operations by the admin such as creating, deleting and updating users.
\\
\begin{itemize}
\item GetAll (GET) - Admin can access Users.
\item DeleteByID (DELETE) - Deletes a user.
\item Create (POST) - Creates new user.
\item updateType (UPDATE) - Updates users subscription type.\\
\end{itemize}

\textbf{Resource: WorkoutRecord}
\\
Description: WorkoutRecord allows the user to interact and handle their workout data.
\\
\begin{itemize}
\item addRecord (POST) - User adds an exercise entry into their workout.
\item getRecordsByUser (GET) - User gets access and sees date of previous workouts.
\item deleteByID (DELETE) - User removes their workout from their records.
\item getAggregatedStatsByUser (GET) - User can view their stats over last 7 days.
\end{itemize}


\section{Authentication and Authorization}

\subsection{Authentication}
GetStanding implements three user types:
\begin{itemize}
\item Admin 
\item Standard User 
\item Premium User 
\end{itemize}
Login is implemented using encrypted passwords.\\
Every existing user has their own JWT token generated during signing up. The token is verified and if it's valid the request provides access to the user type.
\subsection{Authorization}
Middleware checks user's role and grants access to role-specific elements. For example Admins can create and delete exercises which other users can do. Users which are not logged in can not access the dashboard and the perks of having an account, only being able to browse homepage. Standard User has ability to log their workouts however it is impossible for him to access the meals panel and track their meals which is unlocked by the Premium User. In the near future the app will be expanded adding new features for different users.

\subsection{Role table}
\begin{itemize}[noitemsep]
    \item \textbf{Register and Login}
	\begin{itemize}
	\item Actor: User
	\item Description: A user sings up for an account and logs in. 
	\end{itemize}
    \item \textbf{Meals Tracking}
	\begin{itemize}
	\item Actor: Premium User
	\item Description: A user adds meals to the system and data is reflected on addition to show users' consumption. 
	\end{itemize}
    \item \textbf{Workout Tracking}
	\begin{itemize}
	\item Actor: User
	\item Description: A user logs workouts and can view their progress in the app.
	\end{itemize}
    \item \textbf{Adding and Removing}
	\begin{itemize}
	\item Actor: Admin
	\item Description: Admin manages all users, workouts and meals and can create and delete them. 
	\end{itemize}
\end{itemize}
\section{Conclusion}
GetStanding encourages a healthier lifestyle by granting users an application to track their habbits. Thanks to the MVC structure and secure user login, it offers a solid foundation for scalability. The structured models helps keep data organized for easy updates. Future improvements considered by the team include hydration tracking, goal reminders, and social features to help users stay motivated.
\end{document}