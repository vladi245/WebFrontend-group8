\documentclass[12pt]{article}


% PACKAGES

\usepackage[margin=2.5cm]{geometry}
\usepackage{fancyhdr}
\usepackage{graphicx}
\usepackage{array}
\usepackage{hyperref}
\usepackage{titlesec}
\usepackage{tocloft}


% HEADER & FOOTER

\pagestyle{fancy}
\fancyhf{}
\fancyhead[L]{[GetStanding]}
\fancyhead[R]{Semester Project 3 - 2025}
\fancyfoot[C]{\thepage}

\begin{document}


% TITLE PAGE CONTENT 

\begin{center}
    {\Large \textbf{Semester Project 3 - 2025}} \\[1cm]
    {\Huge \textbf{[GetStanding]}} \\[1cm]
    {\Large [Group 8 - Device Name]} \\[1cm]



    \textbf{Authors:}\\[0.5cm]
Piotr Szczesny –piszc24@student.sdu.dk \\
 Vlad-Alexandru Ionita -vlion24@student.sdu.dk \\ 
Karolina Malik -kamal24@student.sdu.dk \\ 
Bartosz Bobkiewicz -babob24@student.sdu.dk \\ 
Viktor Barabás -vibar24@student.sdu.dk \\ 
Azra Talum -aztal24@student.sdu.dk\\[0.5cm]

    \textbf{Supervisor:} 
\end{center}

\newpage



% SHORT HISTORY

\section*{Short History}
[Short history of the device, max ½ page]

\newpage


% ABSTRACT

\section*{Abstract}
[Write your abstract here. Recommended: 150–250 words summarizing purpose, objectives, methods, results, and conclusion.] 

\newpage


% TABLE OF CONTENTS

\tableofcontents
\newpage


% CHAPTERS FROM TEMPLATE


\section{Introduction}
GetStanding is a web based application developed as part of a university course project, based on cases provided by LINAK. It addresses the challenge of maintaining a healthy lifestyle, while working in an office environment.

\subsection{Motivation}
Prolonged sitting can harm both wellbeing and productivity, yet people often forget to stand, stretch, or take short breaks throughout the day. Additionally, the lack of integration between work routines and fitness apps makes it difficult to accurately track and assess physical activity, limiting the ability to maintain a healthy, active lifestyle.

\subsection{Objective}
Our objective is to help people keep track of their habits. The way we do this is through GetStanding, an application that presents user given data in a clear, easy to understand, and intuitive format. Habits like workouts, time spent standing, and meals get tracked on graphs on a weekly basis.  Having the options for a user to select their own height and based on that, through an algorithm, GetStanding gives a recommended standing and sitting height. One of the most important objective for us was to get this done through one application. GetStanding, unlike other fitness tracking applications, also has the capability to connect to a sit and stand desk, to track the user's time spent standing or sitting. Another objective was to make a way for users to keep track of each other through a friends activity feature. This would ensure that a user would return to GetStanding. And lastly, getting rewarded for keeping good habits, such as spending a certain amount of time standing, is also a good incentive. 

\subsection{Delimitation}
Most of our problems were related to the time limitations we had. We wanted to add other features, such as a friends page and rewards, but have decided that we do not have the time capacity to implement them. Our planning was also not always the best, and because of this, some tasks ended up being delayed, making it harder to move forward with the project as a whole. As for the application itself, a limiting factor can be if the user is not constantly keeping track of their workouts, water and food consumption. In this case, the charts will not be able to present the improvement in their habits.

\newpage

\section{Methodology}
[Describe the methods, tools, frameworks used.]

\begin{figure}[h!]
\centering
\rule{10cm}{5cm} % Placeholder for image
\caption{An example figure}
\end{figure}

\subsection{Sub-section}
[Lorem ipsum]

\subsubsection{Sub-sub-section}
[Lorem ipsum]

\newpage

\section{Problem Analysis}
[A detailed breakdown of the problem.]

\subsection{Sub-section}
[Lorem ipsum]

\subsubsection{Sub-sub-section}
[Lorem ipsum]

\newpage

\section{Requirements}
[Functional and non-functional requirements.]

\begin{table}[h!]
\centering
\begin{tabular}{|c|c|c|}
\hline
X & X & X \\ \hline
\end{tabular}
\caption{An example table}
\end{table}

\subsection{Sub-section}
[Lorem ipsum]

\subsubsection{Sub-sub-section}
[Lorem ipsum]

\newpage

\section{Design}
[Architecture, UML diagrams, system components.]

\subsection{Sub-section}
[Lorem ipsum]

\subsubsection{Sub-sub-section}
[Lorem ipsum]

\newpage

\section{Implementation}
[Code, algorithms, libraries used.]

\subsection{Sub-section}
[Lorem ipsum]

\subsubsection{Sub-sub-section}
[Lorem ipsum]

\newpage

\section{Validation}
[Test cases, results, performance analysis.]

\subsection{Sub-section}
[Lorem ipsum]

\subsubsection{Sub-sub-section}
[Lorem ipsum]

\newpage

\section{Conclusion}
[Summary, reflection, objectives achieved.]

\subsection{Summary}
[Lorem ipsum]

\subsection{Future Work}
[Areas for improvement.]

\newpage


% REFERENCES

\section*{References}
There are no sources in the current document.

Follow IEEE citation format as described in the template.

\newpage


% APPENDIX

\section*{Appendix A}
[Supplementary content, raw data, code snippets, logs, manuals, etc.]

\end{document}