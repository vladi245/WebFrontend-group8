\documentclass[12pt]{article}


% PACKAGES

\usepackage[margin=2.5cm]{geometry}
\usepackage{fancyhdr}
\usepackage{graphicx}
\usepackage{array}
\usepackage{hyperref}
\usepackage{titlesec}
\usepackage{tocloft}


% HEADER & FOOTER

\pagestyle{fancy}
\fancyhf{}
\fancyhead[L]{[GetStanding]}
\fancyhead[R]{Semester Project 3 - 2025}
\fancyfoot[C]{\thepage}

\begin{document}


% TITLE PAGE CONTENT 

\begin{center}
    {\Large \textbf{Semester Project 3 - 2025}} \\[1cm]
    {\Huge \textbf{[GetStanding]}} \\[1cm]
    {\Large [Group 8 - Device Name]} \\[1cm]



    \textbf{Authors:}\\[0.5cm]
Piotr Szczesny –piszc24@student.sdu.dk \\
 Vlad-Alexandru Ionita -vlion24@student.sdu.dk \\ 
Karolina Malik -kamal24@student.sdu.dk \\ 
Bartosz Bobkiewicz -babob24@student.sdu.dk \\ 
Viktor Barabás -vibar24@student.sdu.dk \\ 
Azra Talum -aztal24@student.sdu.dk\\[0.5cm]

    \textbf{Supervisor:} 
\end{center}

\newpage



% SHORT HISTORY

\section*{Short History}
Adjustable desks have become an important part of modern workplaces by reducing the negative effects of sitting too long and increasing the general health and productivity of employees. Many design trends have changed the office environment over the years but recently integrating smart technologies such as IoT have become more popular. By connecting IoT technology with a smart desk, user can enjoy the convenience of pushing a button to achieve better ergonomics, health, and well-being while working at your desk. Our project, in collaboration with LINAK focuses on making the workspace smarter,healtier, and more productive. 

\newpage


% ABSTRACT

\section*{Abstract}
\paragraph{} GetStanding addresses the growing need for healthier work environments by integrating reliable technology into adjustable desk systems provided by LINAK. The primary objective is to improve the functionality and user experience of adjustable desks through a web-based application. (WAITING FOR THE OTHER SECTIONS)

\newpage


% TABLE OF CONTENTS

\tableofcontents
\newpage


% CHAPTERS FROM TEMPLATE


\section{Introduction}
GetStanding is a web based application developed as part of a university course project, based on cases provided by LINAK. It addresses the challenge of maintaining a healthy lifestyle, while working in an office environment.

\subsection{Motivation}
Prolonged sitting can harm both wellbeing and productivity, yet people often forget to stand, stretch, or take short breaks throughout the day. Additionally, the lack of integration between work routines and fitness apps makes it difficult to accurately track and assess physical activity, limiting the ability to maintain a healthy, active lifestyle.

\subsection{Objective}
Our objective is to help people keep track of their habits. The way we do this is through GetStanding, an application that presents user given data in a clear, easy to understand, and intuitive format. Habits like workouts, time spent standing, and meals get tracked on graphs on a weekly basis.  Having the options for a user to select their own height and based on that, through an algorithm, GetStanding gives a recommended standing and sitting height. One of the most important objective for us was to get this done through one application. GetStanding, unlike other fitness tracking applications, also has the capability to connect to a sit and stand desk, to track the user's time spent standing or sitting. Another objective was to make a way for users to keep track of each other through a friends activity feature. This would ensure that a user would return to GetStanding. And lastly, getting rewarded for keeping good habits, such as spending a certain amount of time standing, is also a good incentive. 

\subsection{Delimitation}
Most of our problems were related to the time limitations we had. We wanted to add other features, such as a friends page and rewards, but have decided that we do not have the time capacity to implement them. Our planning was also not always the best, and because of this, some tasks ended up being delayed, making it harder to move forward with the project as a whole. As for the application itself, a limiting factor can be if the user is not constantly keeping track of their workouts, water and food consumption. In this case, the charts will not be able to present the improvement in their habits.

\newpage

\section{Methodology}
[Describe the methods, tools, frameworks used.]

\begin{figure}[h!]
\centering
\rule{10cm}{5cm} % Placeholder for image
\caption{An example figure}
\end{figure}

\subsection{Sub-section}
[Lorem ipsum]

\subsubsection{Sub-sub-section}
[Lorem ipsum]

\newpage

\section{Problem Analysis}
\paragraph{} Many studies have demonstrated that prolonged sedentary behavior increases health risks such as obesity, cardiovascular diseases as well as
decreased productivity. Also beck pain, shoulder stiffness, lower back pain, and mental health issues have also been noted as common reasons for decreased work performance. 
\paragraph{}A study stated that people who predominantly sit at work have a 16 higher risk of mortality from all causes, and a 34\% higher risk of mortality from cardiovascular disease To counteract the increased risk, individuals who sit a lot at work would have to engage in an additional 15 to 30 minutes of physical activity per day to reduce their risk to that of individuals who do not predominantly sit, researchers estimated.
\paragraph{} This project will serve as a preventive health solution, aiming to encourage the users to adopt healthier habits during their workday. GetStanding aims to track and monitor the duration of sitting and provide their standing habit insights over time. By seeing their weekly performance and tracking progress, users stay motivated and consistent. In addition to this, GetStanding encourage the user to stand up or take a break after long sitting periods. 



\subsection{Scenarios}

\paragraph{Scenario A: Adjusting desk height}

\begin{description}
    \item[\textbf{Actor:}] Logged-in user
    \item[\textbf{Goal:}] Change the desk height to a comfortable level.
    \item[\textbf{Context:}] User is on the Desk Details page.
\end{description}

\textbf{Main Flow:}
\begin{enumerate}
    \item The user opens the Desk Details page.
    \item The system displays the current desk height.
    \item The user drags the slider up or down to change the desk height.
    \item The displayed numeric height updates in real time.
    \item The desk moves accordingly to match the selected slider position.
\end{enumerate}

\paragraph{Scenario B: Setting Sitting and Standing Preferences}

\begin{description}
    \item[\textbf{Actor:}] Logged-in user
    \item[\textbf{Goal:}] Configure preferred sitting and standing desk heights manually.
    \item[\textbf{Context:}] The user is on the Desk Details page and wants to customize their position settings.
\end{description}

\textbf{Main Flow:}
\begin{enumerate}
    \item The user uses the toggle switch to select either "Sitting" or "Standing" mode.
    \item Under "Preferred Sitting Height (cm) – Manual" or "Preferred Standing Height (cm) – Manual", 
          the user presses the " + " or " - " buttons to adjust the numeric height value.
    \item The system updates the displayed value with each button press.
    \item The user clicks "Confirm" to save their new height setting.
    \item A confirmation message is displayed to indicate successful saving.
    \item The system stores the preferred sitting or standing height in the database.
\end{enumerate}

\paragraph{Scenario 3: Desk Connection }

\begin{description}
    \item[\textbf{Actor:}] Logged-in user
    \item[\textbf{Goal:}] Understand why the desk cannot be controlled and receive clear feedback.
    \item[\textbf{Context:}] The Desk Details page is open, but the system shows a red "Disconnected" status banner.
\end{description}

\textbf{Main Flow:}
\begin{enumerate}
    \item The user opens the Desk Details page.
    \item The system attempts to connect to the desk simulator but fails.
    \item A red banner appears displaying "Disconnected".
    \item The user checks their network or simulator connection and refreshes the page. 
    \item The system reconnects successfully, and displays a green "Connected" banner.
\end{enumerate}

\newpage
\subsection{Key Challenges}
\paragraph{} The most important challenge with this project is to get the user to follow up on their daily statistics on standing as well as to make changes to their daily behavior by providing encouragement. The process of developing new habits is based on the amount of time and effort it takes to develop new habits therefore, many users may ignore the daily stats if they are too busy or if they lack the motivation. This may decrease the effectiveness of the proposed solution. 
\paragraph{} There are technical challenges in this project related to synchronizing data in real-time from the desk sensor, the user application, and the database. Synchronization delays or connectivity issues may prevent tracking stats from being sent, and thus decrease the overall effectiveness of the system. Addressing these challenges strategically will ensure that GetStanding effectively promotes a healthier and more productive workspace environment.

\subsubsection{Sub-sub-section}
[Lorem ipsum]

\newpage

\section{Requirements}
[Functional and non-functional requirements.]

\begin{table}[h!]
\centering
\begin{tabular}{|c|c|c|}
\hline
X & X & X \\ \hline
\end{tabular}
\caption{An example table}
\end{table}

\subsection{Sub-section}
[Lorem ipsum]

\subsubsection{Sub-sub-section}
[Lorem ipsum]

\newpage

\section{Design}
[Architecture, UML diagrams, system components.]

\subsection{Sub-section}
[Lorem ipsum]

\subsubsection{Sub-sub-section}
[Lorem ipsum]

\newpage

\section{Implementation}
[Code, algorithms, libraries used.]

\subsection{Sub-section}
[Lorem ipsum]

\subsubsection{Sub-sub-section}
[Lorem ipsum]

\newpage

\section{Validation}
[Test cases, results, performance analysis.]

\subsection{Sub-section}
[Lorem ipsum]

\subsubsection{Sub-sub-section}
[Lorem ipsum]

\newpage

\section{Conclusion}
[Summary, reflection, objectives achieved.]

\subsection{Summary}
[Lorem ipsum]

\subsection{Future Work}
[Areas for improvement.]

\newpage


% REFERENCES

\section*{References}
There are no sources in the current document.

Follow IEEE citation format as described in the template.

\newpage


% APPENDIX

\section*{Appendix A}
[Supplementary content, raw data, code snippets, logs, manuals, etc.]

\end{document}