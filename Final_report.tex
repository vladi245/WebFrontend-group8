\documentclass[12pt]{article}


% PACKAGES

\usepackage[margin=2.5cm]{geometry}
\usepackage{fancyhdr}
\usepackage{graphicx}
\usepackage{array}
\usepackage{hyperref}
\usepackage{titlesec}
\usepackage{tocloft}


% HEADER & FOOTER

\pagestyle{fancy}
\fancyhf{}
\fancyhead[L]{[GetStanding]}
\fancyhead[R]{Semester Project 3 - 2025}
\fancyfoot[C]{\thepage}

\begin{document}


% TITLE PAGE CONTENT 

\begin{center}
    {\Large \textbf{Semester Project 3 - 2025}} \\[1cm]
    {\Huge \textbf{[GetStanding]}} \\[1cm]
    {\Large [Group 8 - Device Name]} \\[1cm]



    \textbf{Authors:}\\[0.5cm]
Piotr Szczesny –piszc24@student.sdu.dk \\
 Vlad-Alexandru Ionita -vlion24@student.sdu.dk \\ 
Karolina Malik -kamal24@student.sdu.dk \\ 
Bartosz Bobkiewicz -babob24@student.sdu.dk \\ 
Viktor Barabás -vibar24@student.sdu.dk \\ 
Azra Talum -aztal24@student.sdu.dk\\[0.5cm]

    \textbf{Supervisor:} 
\end{center}

\newpage



% SHORT HISTORY

\section*{Short History}
Adjustable desks have become an important part of modern workplaces by reducing the negative effects of sitting too long and increasing the general health and productivity of employees. Many design trends have changed the office environment over the years but recently integrating smart technologies such as IoT have become more popular. By connecting IoT technology with a smart desk, user can enjoy the convenience of pushing a button to achieve better ergonomics, health, and well-being while working at your desk. Our project, in collaboration with LINAK focuses on making the workspace smarter,healtier, and more productive. 

\newpage


% ABSTRACT

\section*{Abstract}
\paragraph{} GetStanding addresses the growing need for healthier work environments by integrating reliable technology into adjustable desk systems provided by LINAK. The primary objective is to improve the functionality and user experience of adjustable desks through a web-based application. (WAITING FOR THE OTHER SECTIONS)

\newpage


% TABLE OF CONTENTS

\tableofcontents
\newpage


% CHAPTERS FROM TEMPLATE


\section{Introduction}
GetStanding is a web based application developed as part of a university course project, based on cases provided by LINAK. It addresses the challenge of maintaining a healthy lifestyle, while working in an office environment.

\subsection{Motivation}
Prolonged sitting can harm both wellbeing and productivity, yet people often forget to stand, stretch, or take short breaks throughout the day. Additionally, the lack of integration between work routines and fitness apps makes it difficult to accurately track and assess physical activity, limiting the ability to maintain a healthy, active lifestyle.

\subsection{Objective}
Our objective is to help people keep track of their habits. The way we do this is through GetStanding, an application that presents user given data in a clear, easy to understand, and intuitive format. Habits like workouts, time spent standing, and meals get tracked on graphs on a weekly basis.  Having the options for a user to select their own height and based on that, through an algorithm, GetStanding gives a recommended standing and sitting height. One of the most important objective for us was to get this done through one application. GetStanding, unlike other fitness tracking applications, also has the capability to connect to a sit and stand desk, to track the user's time spent standing or sitting. Another objective was to make a way for users to keep track of each other through a friends activity feature. This would ensure that a user would return to GetStanding. And lastly, getting rewarded for keeping good habits, such as spending a certain amount of time standing, is also a good incentive. 

\subsection{Delimitation}
Most of our problems were related to the time limitations we had. We wanted to add other features, such as a friends page and rewards, but have decided that we do not have the time capacity to implement them. Our planning was also not always the best, and because of this, some tasks ended up being delayed, making it harder to move forward with the project as a whole. As for the application itself, a limiting factor can be if the user is not constantly keeping track of their workouts, water and food consumption. In this case, the charts will not be able to present the improvement in their habits.

\newpage



\section{Methodology}

\subsection{Development Process and Collaboration}
Based on the knowledge the team has gained from as early as the first semester of software engineering studies and the hands-on experience from previous projects, following an Agile methodology, Scrum to be specific, was the obvious choice.\\\\
The development process was split into two-week sprints, with Jira used as a tool for organization and keeping all team members on track. To ensure smooth development, each sprint had clear objectives that were decided during sprint planning meetings. Throughout those sessions, the backlog was populated with work items, which were assigned to adequate epics and split between team members.\\
During those two weeks the team combined online communication with in-person meetings, which frequency varied based on current needs. \\
After each sprint finished, sprint review sessions were held. During those meetings the team could reflect on the progress that was made. They were essential for understanding how much workload is realistic to complete in the span of two weeks, which was crucial for improving upcoming development and avoiding setting unrealistic expectations.\\\\
In practice the process of developing GetStanding was iterative. Starting from early sprints that focused on establishing the base structure of the web application, and ending with later sprints that focused on adding additional features and refinements. Moreover, feedback was not limited to formal meetings. Frequent informal conversations, short check-ins, group coding sessions were an integral part of how the project was carried out. This approach made solving encountered issues easier, as it was an opportunity to share knowledge, exchange ideas, and decide on how to move forward together.\\
By combining this gradual way of delivering the application with continuous internal feedback, as well as receiving guidance from the supervisor, and finally demo presentations for LINAK and the project coordinator, the team was able to improve both GetStanding and work practices. 
\subsection{Technology Stack}

\subsubsection{Frontend}

For developing the frontend the team used React with Vite. Keeping the web application consistent across different pages was of great importance, therefore React, which is characterized by its component-based approach aligned with the teams vision. It made creating reusable user interface components possible, which made the development process easier by keeping the code clearer, easier to maintain and avoiding code repetition. Due to choosing Vite, the team was able to work in a fast development environment, since it provided quick feedback.\\
Prior to writing any code, a design was vital, thus Figma was the starting point of every frontend element. The team used a shared folder for all the designs, which also played an important role. Because the team members could see each others' designs, the application achieved a cohesive look, despite having six developers working on it.
\subsubsection{Backend}

The backend was implemented using Node.js with Express.js. This set up was selected due to its simplicity and suitability for developing RESTful APIs. Additionally it is well-established in the field, offers a broad set of libraries and comes with wide community-support.

\subsubsection{Data Storage and Infrastructure}
-postgresgl\\
For containerization purposes the team used Docker. It helped with keeping the environment consistent across different devices and reduced configuration problems. Most of the team had previous experience with mentioned tools from the last semester's project, therefore reusing them was a logical choice, that allowed the team members to build on prior experience.\\\\
One of the core requirements for this project was the integration of a Raspberry Pi Pico W board. The team was introduced to it during the Programming for Hardware Constrained Environments course, thus its use in the project provided an opportunity to apply the concepts learned there. Even though the Pico was not strictly required for the main idea of GetStanding to work, it was included to demonstrate how a small device can become a part of something bigger, in this case a larger system, and still play a meaningful role.

\subsection{Version Control}
Git and GitHub were the tools used throughout the whole development of GetStanding. To keep the codebase stable, work was carried out in separate branches rather than the main branch. In order to make the review process more manageable the team tried to stick to suitable and clear naming habits, that includes branches, commits, pull requests. GitHub was not only used to store code, but also as a way of receiving feedback. Each pull request was reviewed by at least one other member and comments were left if necessary. Also, the reviewing focused not only on whether the code was working, but also examined it from a maintainability and readability perspective.\\
Collaborating this way and mutual support improved the team's spirit and contributed to the growth as developers.\\\\
Overall, the selected way of working, tools, and technologies, combined with the experience from previous semesters and the new knowledge acquired this term, is what made designing and implementing GetStanding possible.


\newpage

\section{Problem Analysis}
\paragraph{} Many studies have demonstrated that prolonged sedentary behavior increases health risks such as obesity, cardiovascular diseases as well as
decreased productivity. Also beck pain, shoulder stiffness, lower back pain, and mental health issues have also been noted as common reasons for decreased work performance. 
\paragraph{}A study stated that people who predominantly sit at work have a 16 higher risk of mortality from all causes, and a 34\% higher risk of mortality from cardiovascular disease To counteract the increased risk, individuals who sit a lot at work would have to engage in an additional 15 to 30 minutes of physical activity per day to reduce their risk to that of individuals who do not predominantly sit, researchers estimated.
\paragraph{} This project will serve as a preventive health solution, aiming to encourage the users to adopt healthier habits during their workday. GetStanding aims to track and monitor the duration of sitting and provide their standing habit insights over time. By seeing their weekly performance and tracking progress, users stay motivated and consistent. In addition to this, GetStanding encourage the user to stand up or take a break after long sitting periods. 



\subsection{Scenarios}

\paragraph{Scenario A: Adjusting desk height}

\begin{description}
    \item[\textbf{Actor:}] Logged-in user
    \item[\textbf{Goal:}] Change the desk height to a comfortable level.
    \item[\textbf{Context:}] User is on the Desk Details page.
\end{description}

\textbf{Main Flow:}
\begin{enumerate}
    \item The user opens the Desk Details page.
    \item The system displays the current desk height.
    \item The user drags the slider up or down to change the desk height.
    \item The displayed numeric height updates in real time.
    \item The desk moves accordingly to match the selected slider position.
\end{enumerate}

\paragraph{Scenario B: Setting Sitting and Standing Preferences}

\begin{description}
    \item[\textbf{Actor:}] Logged-in user
    \item[\textbf{Goal:}] Configure preferred sitting and standing desk heights manually.
    \item[\textbf{Context:}] The user is on the Desk Details page and wants to customize their position settings.
\end{description}

\textbf{Main Flow:}
\begin{enumerate}
    \item The user uses the toggle switch to select either "Sitting" or "Standing" mode.
    \item Under "Preferred Sitting Height (cm) – Manual" or "Preferred Standing Height (cm) – Manual", 
          the user presses the " + " or " - " buttons to adjust the numeric height value.
    \item The system updates the displayed value with each button press.
    \item The user clicks "Confirm" to save their new height setting.
    \item A confirmation message is displayed to indicate successful saving.
    \item The system stores the preferred sitting or standing height in the database.
\end{enumerate}

\paragraph{Scenario 3: Desk Connection }

\begin{description}
    \item[\textbf{Actor:}] Logged-in user
    \item[\textbf{Goal:}] Understand why the desk cannot be controlled and receive clear feedback.
    \item[\textbf{Context:}] The Desk Details page is open, but the system shows a red "Disconnected" status banner.
\end{description}

\textbf{Main Flow:}
\begin{enumerate}
    \item The user opens the Desk Details page.
    \item The system attempts to connect to the desk simulator but fails.
    \item A red banner appears displaying "Disconnected".
    \item The user checks their network or simulator connection and refreshes the page. 
    \item The system reconnects successfully, and displays a green "Connected" banner.
\end{enumerate}

\newpage
\subsection{Key Challenges}
\paragraph{} The most important challenge with this project is to get the user to follow up on their daily statistics on standing as well as to make changes to their daily behavior by providing encouragement. The process of developing new habits is based on the amount of time and effort it takes to develop new habits therefore, many users may ignore the daily stats if they are too busy or if they lack the motivation. This may decrease the effectiveness of the proposed solution. 
\paragraph{} There are technical challenges in this project related to synchronizing data in real-time from the desk sensor, the user application, and the database. Synchronization delays or connectivity issues may prevent tracking stats from being sent, and thus decrease the overall effectiveness of the system. Addressing these challenges strategically will ensure that GetStanding effectively promotes a healthier and more productive workspace environment.

\subsubsection{Sub-sub-section}
[Lorem ipsum]

\newpage

\section{Requirements}
This section represents the functional and non-functional requirements of the GetStanding application. Functional requirements describe the core requirements that the application must respect, while non-functional requirements focus on the performance, security and usability of the application, rather than what it is supposed to do. This set of requirements guides the development of the application to guarantee that it meets user needs.
\subsection*{Functional Requirements}
\begin{itemize}
	\item Users shall be able to sign up and log in to the application.
	\item Users shall be able to update their profile information, including username and height.
	\item The UI shall allow input and editing of user data.
	\item The system shall store and retrieve user information and desk preferrence data in a database.
	\item The system shall log desk usage data, including sitting and standing durations.
	\item The system shall generate health recommendations based on the tracked data.
	\item The application should provide the user with data regarding their sitting time using graphs.
	\item The system shall be able to control the desk’s height via the WIFI2BLE Box Web API.
	\item The system shall retrieve and display the current desk height and position.
	 \item Users shall be able to customize their desk settings, such as preferred standing and sitting heights.
\end{itemize}

\subsection*{Non-Functional Requirements}

\begin{itemize}
    \item The system shall support multiple users.
    \item The User Interface shall be intuitive and user-friendly.
    \item User data shall be protected with secure authentication.
    \item The app shall be compatible with major modern browsers.
    \item The system shall provide clear, user-friendly error messages (in plain English).
\end{itemize}

\newpage

\section{Design}
[Architecture, UML diagrams, system components.]

\subsection{Sub-section}
[Lorem ipsum]

\subsubsection{Sub-sub-section}
[Lorem ipsum]

\newpage

\section{Implementation}
[Code, algorithms, libraries used.]

\subsection{Sub-section}
[Lorem ipsum]

\subsubsection{Sub-sub-section}
[Lorem ipsum]

\newpage

\section{Validation}
[Test cases, results, performance analysis.]

\subsection{MoSCoW}
\subsection{Must Have}
\begin{itemize}
    \item User can log in to their own account.
    \item User can sign up and create a new account.
    \item User can remove their account and information from the app.
    \item User can connect the platform to the desk.
    \item User can control the desk height from the interface.
    \item User can log out from their account.
    \item User cannot access the screens of the app when logged out.
    \item App must have multiple user types.
\end{itemize}
\subsection{Should Have}
\begin{itemize}
    \item Endpoints secured from unauthorized access.
    \item User can change their name.
    \item User can add and log their meals and workouts.
    \item User can see their own standing stats.
\end{itemize}
\subsection{Could Have}
\begin{itemize}
    \item User could add friends and see their feed.
    \item User could receive email notifications about their progress.
    \item App could be hosted on a VPS.
    \item User could change their email and password.
\end{itemize}
\subsection{Won't Have}
\begin{itemize}
    \item App won't have a reward system.
    \item App won't accept payments.
    \item App won't have a newsletter.
\end{itemize}


\subsection{Sub-section}
[Lorem ipsum]

\subsubsection{Sub-sub-section}
[Lorem ipsum]

\newpage

\section{Conclusion}
GetStanding was developed as a semester project in collaboration with LINAK, based on the scenarios given by them at the start of the sememster. The team chose to make an application to help people, with a focus on office workers, keep track of their habits. Despite some changes from the original plan for the application, the main functions were successfully implemented such as tracking of calorie intake, workouts and water consumption. The most important feature is that GetStanding can connect to a sit and stand desk, helping the user keep track of the time they spend standing throughout the day. While there were challanges throughout the developmet proces, such as delays that ended up affecting the teams progression. The communication inside the team was wery friendly most of time not serious even, this also affected our efficieny  For some of the team members this was the (outside of classes) first time working with multiple repositories and embedded elements like the Pico board, which added to the level of difficulty. 


\subsection{Summary}
Overall, it can be said that the outcome of the project was a success. GetStanding has provided us with valuable experince both in technical and teamwork scenarios. This will be a good foundation to build on in the future.

\subsection{Future Work}
As for future improvements, functionalities that not have been added would be great improvements. For example the friends page,notifications or even a rewards system. These functions specifically would be really good to retain users for the website. As for the team, working on making things happen on time should be the first priority. Having better communication could be the key to this, so in the next semester this should be something everyone aims for.

\newpage


% REFERENCES

\section*{References}
There are no sources in the current document.

Follow IEEE citation format as described in the template.

\newpage


% APPENDIX

\section*{Appendix A}
[Supplementary content, raw data, code snippets, logs, manuals, etc.]

\end{document}