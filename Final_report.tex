\documentclass[12pt]{article}


% PACKAGES

\usepackage[margin=2.5cm]{geometry}
\usepackage{fancyhdr}
\usepackage{graphicx}
\usepackage{array}
\usepackage{hyperref}
\usepackage{titlesec}
\usepackage{tocloft}


% HEADER & FOOTER

\pagestyle{fancy}
\fancyhf{}
\fancyhead[L]{[GetStanding]}
\fancyhead[R]{Semester Project 3 - 2025}
\fancyfoot[C]{\thepage}

\begin{document}


% TITLE PAGE CONTENT 

\begin{center}
    {\Large \textbf{Semester Project 3 - 2025}} \\[1cm]
    {\Huge \textbf{[GetStanding]}} \\[1cm]
    {\Large [Group 8 - Device Name]} \\[1cm]



    \textbf{Authors:}\\[0.5cm]
Piotr Szczesny –piszc24@student.sdu.dk \\
 Vlad-Alexandru Ionita -vlion24@student.sdu.dk \\ 
Karolina Malik -kamal24@student.sdu.dk \\ 
Bartosz Bobkiewicz -babob24@student.sdu.dk \\ 
Viktor Barabás -vibar24@student.sdu.dk \\ 
Azra Talum -aztal24@student.sdu.dk\\[0.5cm]

    \textbf{Supervisor:} 
\end{center}

\newpage



% SHORT HISTORY

\section*{Short History}
Adjustable desks have become an important part of modern workplaces by reducing the negative effects of sitting too long and increasing the general health and productivity of employees. Many design trends have changed the office environment over the years but recently integrating smart technologies such as IoT have become more popular. By connecting IoT technology with a smart desk, user can enjoy the convenience of pushing a button to achieve better ergonomics, health, and well-being while working at your desk. Our project, in collaboration with LINAK focuses on making the workspace smarter,healtier, and more productive. 

\newpage


% ABSTRACT

\section*{Abstract}
\paragraph{} GetStanding addresses the growing need for healthier work environments by integrating reliable technology into adjustable desk systems provided by LINAK. The primary objective is to improve the functionality and user experience of adjustable desks through a web-based application. (WAITING FOR THE OTHER SECTIONS)

\newpage


% TABLE OF CONTENTS

\tableofcontents
\newpage


% CHAPTERS FROM TEMPLATE


\section{Introduction}
GetStanding is a web based application developed as part of a university course project, based on cases provided by LINAK. It addresses the challenge of maintaining a healthy lifestyle, while working in an office environment.

\subsection{Motivation}
Prolonged sitting can harm both wellbeing and productivity, yet people often forget to stand, stretch, or take short breaks throughout the day. Additionally, the lack of integration between work routines and fitness apps makes it difficult to accurately track and assess physical activity, limiting the ability to maintain a healthy, active lifestyle.

\subsection{Objective}
Our objective is to help people keep track of their habits. The way we do this is through GetStanding, an application that presents user given data in a clear, easy to understand, and intuitive format. Habits like workouts, time spent standing, and meals get tracked on graphs on a weekly basis.  Having the options for a user to select their own height and based on that, through an algorithm, GetStanding gives a recommended standing and sitting height. One of the most important objective for us was to get this done through one application. GetStanding, unlike other fitness tracking applications, also has the capability to connect to a sit and stand desk, to track the user's time spent standing or sitting. Another objective was to make a way for users to keep track of each other through a friends activity feature. This would ensure that a user would return to GetStanding. And lastly, getting rewarded for keeping good habits, such as spending a certain amount of time standing, is also a good incentive. 

\subsection{Delimitation}
Most of our problems were related to the time limitations we had. We wanted to add other features, such as a friends page and rewards, but have decided that we do not have the time capacity to implement them. Our planning was also not always the best, and because of this, some tasks ended up being delayed, making it harder to move forward with the project as a whole. As for the application itself, a limiting factor can be if the user is not constantly keeping track of their workouts, water and food consumption. In this case, the charts will not be able to present the improvement in their habits.

\newpage



\section{Methodology}

\subsection{Development Process and Collaboration}
Based on the knowledge the team has gained from as early as the first semester of software engineering studies and the hands-on experience from previous projects, following an Agile methodology, Scrum to be specific, was the obvious choice.\\\\
The development process was split into two-week sprints, with Jira used as a tool for organization and keeping all team members on track. To ensure smooth development, each sprint had clear objectives that were decided during sprint planning meetings. Throughout those sessions, the backlog was populated with work items, which were assigned to adequate epics and split between team members.\\
During those two weeks the team combined online communication with in-person meetings, which frequency varied based on current needs. \\
After each sprint finished, sprint review sessions were held. During those meetings the team could reflect on the progress that was made. They were essential for understanding how much workload is realistic to complete in the span of two weeks, which was crucial for improving upcoming development and avoiding setting unrealistic expectations.  

\subsection{Technology Stack}

\subsubsection{Frontend}

For developing the frontend the team used React with Vite. Keeping the web application consistent across different pages was of great importance, therefore React, which is characterized by its component-based approach aligned with the teams vision. It made creating reusable user interface components possible, which made the development process easier by keeping the code clearer, easier to maintain and avoiding code repetition. Due to choosing Vite, the team was able to work in a fast development environment, since it provided quick feedback.\\
Prior to writing any code, a design was vital, thus Figma was the starting point of every frontend element. The team used a shared folder for all the designs, which also played an important role. Because the team members could see each others' designs, the application achieved a cohesive look, despite having six developers working on it.

\subsubsection{Backend}

The backend was implemented using Node.js with Express.js.

\subsubsection{Data Storage and Infrastructure}
-postgresgl
-docker
-pico board

\subsection{Version Control}
Git and GitHub were the tools used throughout the whole development of GetStanding. To keep the codebase stable, work was carried out in seperate branches rather than the main branch. In order to make the review process more manageable the team tried to stick to suitable and clear naming habits, that includes branches, commits, pull requests. Each pull request was reviewd by at least one other member and comments were left if necessary.\\
Collaborating this way and mutual support improved the team's spirit and contributed to the growth as developers.



\newpage

\section{Problem Analysis}
\paragraph{} Many studies have demonstrated that prolonged sedentary behavior increases health risks such as obesity, cardiovascular diseases as well as
decreased productivity. Also beck pain, shoulder stiffness, lower back pain, and mental health issues have also been noted as common reasons for decreased work performance. 
\paragraph{}A study stated that people who predominantly sit at work have a 16 higher risk of mortality from all causes, and a 34\% higher risk of mortality from cardiovascular disease To counteract the increased risk, individuals who sit a lot at work would have to engage in an additional 15 to 30 minutes of physical activity per day to reduce their risk to that of individuals who do not predominantly sit, researchers estimated.
\paragraph{} This project will serve as a preventive health solution, aiming to encourage the users to adopt healthier habits during their workday. GetStanding aims to track and monitor the duration of sitting and provide their standing habit insights over time. By seeing their weekly performance and tracking progress, users stay motivated and consistent. In addition to this, GetStanding encourage the user to stand up or take a break after long sitting periods. 

\subsection{Key Challenges}
\paragraph{} The most important challenge with this project is to get the user to follow up on their daily statistics on standing as well as to make changes to their daily behavior by providing encouragement. The process of developing new habits is based on the amount of time and effort it takes to develop new habits therefore, many users may ignore the daily stats if they are too busy or if they lack the motivation. This may decrease the effectiveness of the proposed solution. 
\paragraph{} There are technical challenges in this project related to synchronizing data in real-time from the desk sensor, the user application, and the database. Synchronization delays or connectivity issues may prevent tracking stats from being sent, and thus decrease the overall effectiveness of the system. Addressing these challenges strategically will ensure that GetStanding effectively promotes a healthier and more productive workspace environment.

\subsubsection{Sub-sub-section}
[Lorem ipsum]

\newpage

\section{Requirements}
This section represents the functional and non-functional requirements of the GetStanding application. Functional requirements describe the core requirements that the application must respect, while non-functional requirements focus on the performance, security and usability of the application, rather than what it is supposed to do. This set of requirements guides the development of the application to guarantee that it meets user needs.
\subsection*{Functional Requirements}
\begin{itemize}
	\item Users shall be able to sign up and log in to the application.
	\item Users shall be able to update their profile information, including username and height.
	\item The UI shall allow input and editing of user data.
	\item The system shall store and retrieve user information and desk preferrence data in a database.
	\item The system shall log desk usage data, including sitting and standing durations.
	\item The system shall generate health recommendations based on the tracked data.
	\item The application should provide the user with data regarding their sitting time using graphs.
	\item The system shall be able to control the desk’s height via the WIFI2BLE Box Web API.
	\item The system shall retrieve and display the current desk height and position.
	 \item Users shall be able to customize their desk settings, such as preferred standing and sitting heights.
\end{itemize}

\subsection*{Non-Functional Requirements}

\begin{itemize}
    \item The system shall support multiple users.
    \item The User Interface shall be intuitive and user-friendly.
    \item User data shall be protected with secure authentication.
    \item The app shall be compatible with major modern browsers.
    \item The system shall provide clear, user-friendly error messages (in plain English).
\end{itemize}

\newpage

\section{Design}
[Architecture, UML diagrams, system components.]

\subsection{Sub-section}
[Lorem ipsum]

\subsubsection{Sub-sub-section}
[Lorem ipsum]

\newpage

\section{Implementation}
The implementation of GetStanding was structured to ensure modularity, scalability, and maintainability. The project was divided into two main parts: the frontend and the backend, each designed with best practices in mind to facilitate collaboration and future development.

\subsection{Frontend}
The frontend for GetStandind is built with React and Vite, resulting in a modular, component-based architecture. This approach made it easy to implement new ideas and quickly pivot by discarding features when necessary. Additionally, Vite’s development server with live refresh significantly improved the developer experience, allowing changes to be reflected instantly without needing to restart the application. On top of that use of Docker in later stages of the project made sure that all deployments were working on the correct versions of packages, and most importantly Node.js \\
The project is split into pages stored in the folder page/ and components stored in the folder components/. The decision for this setup was to make a new page and component creation easier, where pages function as a “component of components” connected to a specific route.

\subsubsection{Routing - src/App.tsx}
Routing is handled using the react-router-dom package, with all routes managed in the App.tsx file. As mentioned earlier, this approach made it easier for the team to add new pages and routes. Additionally, it simplified the setup of a custom /404 page, ensuring lost users are redirected appropriately and do not get lost navigating the app.\\
Centralizing routing enabled the team to implement a ProtectedRoute component, which manages client-side access control for dashboard subpages. Each route wrapped with ProtectedRoute is accessible only after a user logs in. Furthermore, this component supports a requiredType property, allowing specification of the account type necessary to access certain pages. For example, only an admin can read messages from the contact form, while features such as Meals are available exclusively to premium users of GetStanding.

\subsubsection{Routing - src/App.tsx}
Routing is handled using the react-router-dom package, with all routes managed in the App.tsx file. As mentioned earlier, this approach made it easier for the team to add new pages and routes. Additionally, it simplified the setup of a custom /404 page, ensuring lost users are redirected appropriately and do not get lost navigating the app.\\
Centralizing routing enabled the team to implement a ProtectedRoute component, which manages client-side access control for dashboard subpages. Each route wrapped with ProtectedRoute is accessible only after a user logs in. Furthermore, this component supports a requiredType property, allowing specification of the account type necessary to access certain pages. For example, only an admin can read messages from the contact form, while features such as Meals are available exclusively to premium users of GetStanding.

\subsubsection{services/api.ts}
During development, the team recognized the need to centralize API calls. Previously, API requests were scattered across pages and components, leading to inconsistencies, especially when system ports were occupied on different operating systems. To address this, the team created a single entry point for all API connections in services/api.ts. This file stores the backend server address, making it easier to update API endpoints and ensuring that the JWT token (stored in local storage) is automatically included with every request made from the frontend.

\subsubsection{Styling - .module.css files}
The team opted not to use CSS frameworks like Tailwind. Instead, they chose to use separate CSS files for each component and page. To avoid conflicts and maintain flexibility, CSS Modules were employed. This approach allows for repeating class names across components without unintended overrides, ensuring a clean and maintainable styling system.

\subsection{Backend}
The backend was designed around the MVC (Model-View-Controller) architecture and a RESTful API. The team initially focused on developing the frontend with mock data to better understand the system’s requirements before diving into complex business logic. 

\subsubsection{Routes}
Similar to the frontend, the backend centralises route management in the app.js file. However, instead of defining routes directly, it imports them from separate files in the routes/ folder (e.g., routes/desk.js, routes/adminRoutes.js). This modular approach improves code clarity and maintainability, especially given the backend’s higher number of routes compared to the frontend. 

\subsubsection{Models}
In the MVC architecture, Models handle the core business logic and database interactions. They define functions to query the PostgreSQL database, ensuring data integrity and consistency.

\subsubsection{Controllers}
Controllers act as the intermediary between the View (frontend) and the Models. They handle user input, perform validation (checking if a user exists), and manage the flow of data between the frontend and backend.

\subsubsection{Authorisation - JWT}
User authentication is managed using JWT (JSON Web Tokens). Upon login, users receive a JWT token with an expiration time of 1 hour. This token is included in every API request to the backend, where it is validated. Requests without a valid token are automatically rejected, ensuring secure access to the system.

The token is generated in src/middleware/authMiddleware.js and imported into every API endpoint as authMiddleware.

\subsection{Database}
\subsubsection{Functions}
To ensure scalability and future-proofing, the team implemented PostgreSQL functions as a core part of the database architecture. These functions are designed to encapsulate specific business logic and are called directly from the backend models. This approach not only modularizes database operations but also enhances security by allowing granular control over access permissions. Instead of granting users broad access to entire tables, the team can now restrict permissions to individual functions, minimizing the risk of unauthorized data manipulation or exposure.\\
During development, this strategy provided greater control and flexibility. By abstracting complex queries and operations into reusable functions, the team reduced redundancy, improved code maintainability, and streamlined debugging. Additionally, it allowed for consistent data handling, as all interactions with the database are funneled through these predefined functions, ensuring compliance to business rules and reducing the likelihood of errors. This design ensures that the system can easily accommodate future extensions, such as adding new features or integrating third-party services, without requiring significant refactoring or altering the access levels in the database (creating database users).



\section{Validation}
[Test cases, results, performance analysis.]

\subsection{Sub-section}
[Lorem ipsum]

\subsubsection{Sub-sub-section}
[Lorem ipsum]

\newpage

\section{Conclusion}
GetStanding was developed as a semester project in collaboration with LINAK, based on the scenarios given by them at the start of the sememster. The team chose to make an application to help people, with a focus on office workers, keep track of their habits. Despite some changes from the original plan for the application, the main functions were successfully implemented such as tracking of calorie intake, workouts and water consumption. The most important feature is that GetStanding can connect to a sit and stand desk, helping the user keep track of the time they spend standing throughout the day. While there were challanges throughout the developmet proces, such as delays that ended up affecting the teams progression. The communication inside the team was wery friendly most of time not serious even, this also affected our efficieny  For some of the team members this was the (outside of classes) first time working with multiple repositories and embedded elements like the Pico board, which added to the level of difficulty. 


\subsection{Summary}
Overall, it can be said that the outcome of the project was a success. GetStanding has provided us with valuable experince both in technical and teamwork scenarios. This will be a good foundation to build on in the future.

\subsection{Future Work}
As for future improvements, functionalities that not have been added would be great improvements. For example the friends page,notifications or even a rewards system. These functions specifically would be really good to retain users for the website. As for the team, working on making things happen on time should be the first priority. Having better communication could be the key to this, so in the next semester this should be something everyone aims for.

\newpage


% REFERENCES

\section*{References}
There are no sources in the current document.

Follow IEEE citation format as described in the template.

\newpage


% APPENDIX

\section*{Appendix A}
[Supplementary content, raw data, code snippets, logs, manuals, etc.]

\end{document}